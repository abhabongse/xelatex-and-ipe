%!TEX root = ../main.tex

\section{TEMPLATE FEATURES}

This \LaTeX{} document template has the following features.

\begin{enumerate}

\item Supports text in both Latin and Thai scripts. For example,
    \begin{center}
        สามารถใช้งานภาษาไทยได้ตามปกติ
    \end{center}
    The revised \textit{Sarabun}\sidenote{The font was designed by Suppakit Chalermlarp
    and revised by Cadson Demak under Google 13+1 Thai Fonts Project.}
    font is used as the default font.

\item Math equations also uses the same font by default, e.g.,
    \begin{equation}
        x^2 + y^2 + z^2 = 1 \tag{3-Dimensional Sphere}
    \end{equation}
    \begin{equation}
        x = \frac{-b \pm \sqrt{b^2 - 4ac}}{2a} \tag{Quadratic Solution}
    \end{equation}

\item Supports sidenote \lstinline"\sidenote{As shown here and above.}"\sidenote{As shown here and above.} \\
    Sidenotes are ragged right due to the command \lstinline"\adjustsidenote{\raggedright}"
    at the document preamble (see source).

    If you are not going to use \lstinline"\sidenote" then you may want to 
    re-adjust page margins using \lstinline{geometry} package at the document preamble. 
    
\item Supports simple commands to switch between eight font weights on-the-fly.
    \begin{center}
        \begin{tabular}{rl}
            \lstinline"\thinfont"            & {\thinfont ฟอนต์ชุด Sarabun-Thin}                   \\
            \lstinline"\lightfont"           & {\lightfont ฟอนต์ชุด Sarabun-ExtraLight}            \\
            \lstinline"\normalfont"          & {\normalfont ฟอนต์ชุด Sarabun-Light}                \\
            \lstinline"\heavyfont"           & {\heavyfont ฟอนต์ชุด Sarabun-Regular}               \\
            \lstinline"\thinfont\bfseries"   & {\thinfont\bfseries ฟอนต์ชุด Sarabun-Medium}        \\
            \lstinline"\lightfont\bfseries"  & {\lightfont\bfseries ฟอนต์ชุด Sarabun-SemiBold}     \\
            \lstinline"\normalfont\bfseries" & {\normalfont\bfseries ฟอนต์ชุด Sarabun-Bold}        \\
            \lstinline"\heavyfont\bfseries"  & {\heavyfont\bfseries ฟอนต์ชุด Sarabun-ExtralBold}   \\
        \end{tabular}
    \end{center}
    All fonts shown above also deploy \lstinline"\itshape" variants.

\item A few text decorations such as text pills: \labelpill{LABEL PILL} \labelpillalt{ALT LABEL PILL}
\begin{lstlisting}[numbers=none]
\labelpill{LABEL PILL} \labelpillalt{ALT LABEL PILL}
\end{lstlisting}
    Text highlighting: \rhlt{Red.} \ghlt{Green.} \bhlt{Blue.} \yhlt{Yellow.} \hlt[black!15!white]{And Custom.}
\begin{lstlisting}[numbers=none]
\rhlt{Red.} \ghlt{Green.} \bhlt{Blue.} \yhlt{Yellow.}
\hlt[black!15!white]{And Custom.}
\end{lstlisting}
    Text highlighting also works inside math equations.
    \begin{equation}
        \hlt{x^2} + \hlt{y^2} = z^2  \tag{Pythagorean Theorem}
    \end{equation}

\item Header decorations with the command, e.g., \lstinline"\secbanner[18pc]{NEW SECTION}"
    \begin{center}
        \Large\heavyfont\bfseries\addfontfeature{LetterSpace=10}
        \secbanner[18pc]{NEW SECTION}
    \end{center}
    To use section banner automatically, turn it on with the command \lstinline"\bannerinsec" 
    at the document preamble (result shown in this document uses \lstinline"\bannerinsec[17pc]").

\item Special hair space character \lstinline"\hrsp" which does make huge difference between
    \begin{center}
        \begin{tabular}{rcl}
            Hyphenated-text     & \quad vs \quad &  Hyphenated{\hrsp-\hrsp}text \\
            10 -- 99 \;or\; 10--99  & \quad vs \quad &  10{\hrsp--\hrsp}99 \\
            ต่างๆ นานา           & \quad vs \quad &  ต่าง{\hrsp}ๆ{\hrsp}นานา \\
        \end{tabular}
    \end{center}

\item Free Font Awesome 5 can be included like so: \lstinline"\faShower" = \faShower. \\
    To use the pro version of the font (assume that it is installed in the font directory of the local machine),
    change the following lines in \verb"tenth.sty":
\begin{lstlisting}[numbers=none]
% \RequirePackage[fixed]{fontawesome5}
\RequirePackage[pro,fixed]{fontawesome5}
\end{lstlisting}

\item {\lettersp[15] LETTERSPACING} may be added with the following command:
\begin{lstlisting}[numbers=none]
{\lettersp[5] Hello}  % Turn on letterspacing with value 5
{\lettersp Hello}  % Default letterspacing with value 5
\end{lstlisting}
    The special command \lstinline"\duration" can be used to typeset time duration.
    For example, \lstinline"\duration{FEB 2018 -- JAN 2019}" would produce “\duration{FEB 2018 -- JAN 2019}”.

\item Page header can be customized by redefining \lstinline"\pageheader" command.
    For example, to get the page header you see on this page, put the following code in document preamble:
\begin{lstlisting}[numbers=none]
\renewrobustcmd{\pageheader}{\XeLaTeX{} template}
\end{lstlisting}

\item As you might have seen earlier, source code may be included through \lstinline"lstlisting" package.
    This template pre-defines color palette for source code highlightings (see source).

\begin{lstlisting}[language=python]
import functools
import operator
from typing import Iterable, TypeVar

T = TypeVar['T']


def product(values: Iterable[T]) -> T:
    """
    Compute the product of all given values.
    """
    return functools.reduce(operator.mul, values)
\end{lstlisting}
    Notice that FiraCode font is used, with ligatures enabled 
    (such as ‘\texttt{-{\vphantom{!}}>}’ being substituted by ‘\texttt{->}’).

\item Thai enumeration counters are supported by turning on 
    \lstinline"\redefinethaicounters" in the document preamble.
    
    Then set the \lstinline"label" option of the \lstinline"enumerate"
    to something like \lstinline"label={\thaialphs*.}" or \lstinline"label={\thainum*.}".

    For example, here is the \textbf{list of all known elements in the periodic table}.
    \begin{multicols}{3}
    \begin{enumerate}[label={\thaialphs*.}]
        \item Hydrogen
        \item Helium
        \item Lithium
        \item Beryllium
        \item Boron
        \item Carbon
        \item Nitrogen
        \item Oxygen
        \item Fluorine
        \item Neon
        \item Sodium
        \item Magnesium
        \item Aluminum
        \item Silicon
        \item Phosphorus
        \item Sulfur
        \item Chlorine
        \item Argon
        \item Potassium
        \item Calcium
        \item Scandium
        \item Titanium
        \item Vanadium
        \item Chromium
        \item Manganese
        \item Iron
        \item Cobalt
        \item Nickel
        \item Copper
        \item Zinc
        \item Gallium
        \item Germanium
        \item Arsenic
        \item Selenium
        \item Bromine
        \item Krypton
        \item Rubidium
        \item Strontium
        \item Yttrium
        \item Zirconium
        \item Niobium
        \item Molybdenum
        \item Technetium
        \item Ruthenium
        \item Rhodium
        \item Palladium
        \item Silver
        \item Cadmium
        \item Indium
        \item Tin
        \item Antimony
        \item Tellurium
        \item Iodine
        \item Xenon
        \item Cesium
        \item Barium
        \item Lanthanum
        \item Cerium
        \item Praseodymium
        \item Neodymium
    \end{enumerate}
    \end{multicols}

    \begin{multicols}{3}
    \begin{enumerate}[label={\thaialphs*.},start=62]
        \item Promethium
        \item Samarium
        \item Europium
        \item Gadolinium
        \item Terbium
        \item Dysprosium
        \item Holmium
        \item Erbium
        \item Thulium
        \item Ytterbium
        \item Lutetium
        \item Hafnium
        \item Tantalum
        \item Tungsten
        \item Rhenium
        \item Osmium
        \item Iridium
        \item Platinum
        \item Gold
        \item Mercury
        \item Thallium
        \item Lead
        \item Bismuth
        \item Polonium
        \item Astatine
        \item Radon
        \item Francium
        \item Radium
        \item Actinium
        \item Thorium
        \item Protactinium
        \item Uranium
        \item Neptunium
        \item Plutonium
        \item Americium
        \item Curium
        \item Berkelium
        \item Californium
        \item Einsteinium
        \item Fermium
        \item Mendelevium
        \item Nobelium
        \item Lawrencium
        \item Rutherfordium
        \item Dubnium
        \item Seaborgium
        \item Bohrium
        \item Hassium
        \item Meitnerium
        \item Darmstadtium
        \item Roentgenium
        \item Copernicium
        \item Nihonium
        \item Flerovium
        \item Moscovium
        \item Livermorium
        \item Tennessine
        \item Oganesson
        \item[] \phantom{x}
        \item[] \phantom{x}
    \end{enumerate}
    \end{multicols}

    \textbf{List of Thai months in a year.}
    \begin{enumerate}[label={\thainum*.}]
        \item มกราคม
        \item กุมภาพันธ์
        \item มีนาคม
        \item เมษายน
        \item พฤษภาคม
        \item มิถุนายน
        \item กรกฎาคม
        \item สิงหาคม
        \item กันยายน
        \item ตุลาคม
        \item พฤศจิกายน
        \item ธันวาคม
    \end{enumerate}

\end{enumerate}
